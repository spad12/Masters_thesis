% !TEX TS-program = pdflatex
% !TEX encoding = UTF-8 Unicode

% This is a simple template for a LaTeX document using the "article" class.
% See "book", "report", "letter" for other types of document.

\documentclass[12pt]{article} % use larger type; default would be 10pt

\usepackage[utf8]{inputenc} % set input encoding (not needed with XeLaTeX)

%%% Examples of Article customizations
% These packages are optional, depending whether you want the features they provide.
% See the LaTeX Companion or other references for full information.

%%% PAGE DIMENSIONS
\usepackage{geometry} % to change the page dimensions
\geometry{a4paper} % or letterpaper (US) or a5paper or....
\geometry{margin=0.25in} % for example, change the margins to 2 inches all round
% \geometry{landscape} % set up the page for landscape
%   read geometry.pdf for detailed page layout information

\usepackage{graphicx} % support the \includegraphics command and options

% \usepackage[parfill]{parskip} % Activate to begin paragraphs with an empty line rather than an indent

%%% PACKAGES
\usepackage{booktabs} % for much better looking tables
\usepackage{array} % for better arrays (eg matrices) in maths
\usepackage{paralist} % very flexible & customisable lists (eg. enumerate/itemize, etc.)
\usepackage{verbatim} % adds environment for commenting out blocks of text & for better verbatim
\usepackage{subfig} % make it possible to include more than one captioned figure/table in a single float
% These packages are all incorporated in the memoir class to one degree or another...

\usepackage{hyperref}
\usepackage{wrapfig}
\usepackage{listings}
\lstset{language=C}

%%% HEADERS & FOOTERS
\usepackage{fancyhdr} % This should be set AFTER setting up the page geometry
\pagestyle{fancy} % options: empty , plain , fancy
\renewcommand{\headrulewidth}{0pt} % customise the layout...
\lhead{}\chead{}\rhead{}
\lfoot{}\cfoot{\thepage}\rfoot{}

\usepackage{amsmath}

%%% SECTION TITLE APPEARANCE
%\usepackage{sectsty}
%\allsectionsfont{\sffamily\mdseries\upshape} % (See the fntguide.pdf for font help)
% (This matches ConTeXt defaults)

%%% ToC (table of contents) APPEARANCE
\usepackage[nottoc,notlof,notlot]{tocbibind} % Put the bibliography in the ToC
%\usepackage[titles,subfigure]{tocloft} % Alter the style of the Table of Contents
%\renewcommand{\cftsecfont}{\rmfamily\mdseries\upshape}
%\renewcommand{\cftsecpagefont}{\rmfamily\mdseries\upshape} % No bold!

%%% END Article customizations

%%% The "real" document content comes below...

\title{Report on the Feasibility of Implementing PIC Codes on a GPU}
\author{Joshua Payne}
%\date{} % Activate to display a given date or no date (if empty),
         % otherwise the current date is printed 

\begin{document}

\tableofcontents

\section{Introduction}
	\subsection{Motivation}
		\subsubsection{GPUs vs CPUs}

	\subsection{Overview of sceptic3D}
		\subsubsection{Basic Code Structure}
		\subsubsection{CPU Code Profiling}

	\subsection{Overview of sceptic3Dgpu Goals}
		\subsubsection{Main Routines}
		\subsubsection{Supporting Routines}
		\subsubsection{Challenges to overcome}

\section{Design Options}

	\subsection{Particle List Structure}
		\subsubsection{Other Codes}
		\subsubsection{In house tests}

	\subsection{Charge Assign}
		\subsubsection{Naive Atomic Approach}
		\subsubsection{Other Codes}

	\subsection{Particle List Sort}
		\subsubsection{Costs and Benefits}
		\subsubsection{Other Codes}
		\subsubsection{In house tests}

	\subsection{Particle Advancing}
		\subsubsection{Assumptions}
		\subsubsection{Other Codes}
		\subsubsection{Reinjections and Diagnostics}

	\subsection{Poisson Solve}
		\subsubsection{Desired Performance}
		\subsubsection{Performance vs Implementation Difficulty}

	\subsection{Grid Dimension Constraints and Handling}

\section{Implementation}

	\subsection{Execution Layout of sceptic3D}

	\subsection{Constraining Grid Dimensions}
		\subsubsection{Constraints}
		\subsubsection{Holding to the constraints}

	\subsection{Particle List Transpose}

	\subsection{Particle List Sort}

	\subsection{Charge Assign}

	\subsection{Poisson Solve}

	\subsection{Particle List Advance}

		\subsubsection{Handling Reinjections}

\section{Performance} 

	\subsection{Test Setup}
		\subsubsection{Parameter Space Explored}
		\subsubsection{Machine Parameters}
		\subsubsection{Memory Bandwidth Comparison}
	
	\subsection{Particle list size scan}
	
	\subsection{Grid Size scan}

	\subsection{Kernel Parameters Scan}

\section{Conclusion} 

\end{document}
