% Turing Machine
% Author: Sebastian Sardina
%\documentclass[a4paper,10pt]{article}
%\usepackage[usenames,dvipsnames]{xcolor}
%\usepackage{tikz}
%\usepackage{ifthen}
%\usetikzlibrary{chains,fit,shapes}

%\begin{document}


\noindent
\begin{tikzpicture}
		\tikzstyle{every path}=[very thick]
		\tikzstyle{MainSty}=[->,dashdotted]
		\tikzstyle{CollSty}=[->,color=Emerald,densely dashdotted]
		\tikzstyle{ReinSty}=[->,color=Maroon,dashed]

		\edef\sizetape{1.0cm}
		\tikzstyle{tmtape}=[draw,minimum size=\sizetape]
		\tikzstyle{tmtape1}=[draw,minimum height=\sizetape,minimum width=0.5cm]
		\tikzstyle{tmhead}=[arrow box,draw,minimum size=0.4cm,arrow box
		arrows={east:.25cm}]

		\tikzstyle{cfill1}=[fill=Purple]
		\tikzstyle{cfill2}=[fill=SkyBlue]
		\tikzstyle{cfill3}=[fill=Maroon]
		\tikzstyle{cfill4}=[fill=Emerald]


\def\MainArray{10,9,8,7,6,5,4,3,2,1}
\def\ReinArray{8,5,1}
\def\CollArray{9,5,4,2}

\newcounter{prim@a}
\newcounter{prim@b}
\newcounter{prim@c}
\newcounter{prim@d}


% Figure out which nodes will collide

\begin{scope}[start chain=5,node distance=1cm]

% The initial list
	%\node [on chain=5]	{\begin{tikzpicture}[remember picture]
		\begin{scope}[start chain=1 going below,node distance=-1pt]

		\foreach \i in \MainArray {
			\node [on chain=1,tmtape] (pmain\i) {$\i$};
		}
		\end{scope}
		%\end{tikzpicture}};

% The Collision Check
	%\node [on chain=5]	{\begin{tikzpicture}[remember picture]
		\begin{scope}[xshift=2.0cm,start chain=1 going below,node distance=-1pt]


		\foreach \i in \MainArray{
			
			\foreach \j in \CollArray
			{
				\setcounter{prim@a}{\i}
				\setcounter{prim@b}{\j}
				\ifthenelse{\value{prim@a}=\value{prim@b}}
				{\setcounter{prim@c}{\i}}{}%
			}

			\ifthenelse{\value{prim@a}=\value{prim@c}}
			{ \node [on chain=1,tmtape1,fill=Emerald] (pc\i) {};}%
   		{ \node [on chain=1,tmtape1] (pc\i) {};}%

		}
		\end{scope}
		%\end{tikzpicture}};

% The Reinjection Check
	%\node [on chain=5]	{\begin{tikzpicture}[remember picture]
		\begin{scope}[xshift=3.75cm,start chain=1 going below,node distance=-1pt]

		\foreach \i in \MainArray{

			\foreach \j in \ReinArray
			{
				\setcounter{prim@a}{\i}
				\setcounter{prim@b}{\j}
				\ifthenelse{\value{prim@a}=\value{prim@b}}
				{\setcounter{prim@c}{\i}}{}%
			}

			\foreach \j in \CollArray
			{
				\setcounter{prim@a}{\i}
				\setcounter{prim@b}{\j}
				\ifthenelse{\value{prim@a}=\value{prim@b}}
				{\setcounter{prim@d}{\i}}{}%
			}

			\ifthenelse{\value{prim@a}=\value{prim@c}}
			{ \node [on chain=1,tmtape1,fill=Maroon] (pr\i) {};}%
   		{ \ifthenelse{\value{prim@a}=\value{prim@d}}
				{\node [on chain=1,tmtape1,fill=Emerald] (pr\i) {};}
				{\node [on chain=1,tmtape1] (pr\i) {};}

			}%

		}
		\end{scope}
		%\end{tikzpicture}};

% The initial list
	%\node [on chain=5]	{\begin{tikzpicture}[remember picture,start chain=4 going above,node distance=0.5cm]
	\begin{scope}[xshift=7.5cm,yshift=-9.0cm,start chain=4 going above,node distance=0.5cm]
		\node [on chain=4,tmtape] (wait) {Wait};

		\begin{scope}[yshift=9.0cm,start chain=1 going below,node distance=-1pt]
			\node [on chain=1] {\footnotesize Collide \& Move};

			\foreach \i in \CollArray {

			\foreach \j in \ReinArray
			{
				\setcounter{prim@a}{\i}
				\setcounter{prim@b}{\j}
				\ifthenelse{\value{prim@a}=\value{prim@b}}
				{\setcounter{prim@c}{\i}}{}%
			}

				\ifthenelse{\value{prim@a}=\value{prim@c}}{}
				{\node [on chain=1,tmtape] (lc\i) {$\i$};}%
			}
		\end{scope}

		\begin{scope}[yshift=5.0cm,start chain=1 going below,node distance=-1pt]

			\node [on chain=1] {\footnotesize Reinject \& Move};
			\foreach \i in \ReinArray {
				\node [on chain=1,tmtape] (lr\i) {$\i$};
			}
			\end{scope}

		\end{scope}
		%\end{tikzpicture}};

% The Main list without collision and reinjection
	%\node [on chain=5]	{\begin{tikzpicture}[remember picture]
		\begin{scope}[xshift=11.325cm,start chain=1 going below,node distance=-1pt]

		\foreach \i in \MainArray{

			\foreach \j in \ReinArray
			{
				\setcounter{prim@a}{\i}
				\setcounter{prim@b}{\j}
				\ifthenelse{\value{prim@a}=\value{prim@b}}
				{\setcounter{prim@c}{\i}}{}%
			}

			\foreach \j in \CollArray
			{
				\setcounter{prim@a}{\i}
				\setcounter{prim@b}{\j}
				\ifthenelse{\value{prim@a}=\value{prim@b}}
				{\setcounter{prim@c}{\i}}{}%
			}

			\ifthenelse{\value{prim@a}=\value{prim@c}}
			{ \node [on chain=1,tmtape] (pm\i) {$\i^*$};}%
   		{ \node [on chain=1,tmtape] (pm\i) {$\i$};}%

		}
		\end{scope}
		%\end{tikzpicture}};


		\foreach \i in \MainArray
		{

			\setcounter{prim@a}{\i}

			\foreach \j in \ReinArray
			{
				\setcounter{prim@a}{\i}
				\setcounter{prim@b}{\j}
				\ifthenelse{\value{prim@a}=\value{prim@b}}
				{\setcounter{prim@c}{\i}}{}%
			}

			\foreach \j in \CollArray
			{
				\setcounter{prim@b}{\j}
				\ifthenelse{\value{prim@a}=\value{prim@b}}
				{\setcounter{prim@d}{\i}}{}%
			}

			\ifthenelse{\value{prim@a}=\value{prim@c}}
			{ 
				\draw[MainSty] (pmain\i) to[->] (pc\i);

			  \ifthenelse{\value{prim@a}=\value{prim@d}}
				{\draw[CollSty] (pc\i) to (pr\i);}
				{\draw[MainSty] (pc\i) to (pr\i);}
			  \begin{scope}[ReinSty]
				\draw[->] (pr\i.mid east) to[in=180] (lr\i);
				\draw[->] (lr\i) to[out=0,in=180] (pm\i.mid west);
			  \end{scope}

			}%
   		{\ifthenelse{\value{prim@a}=\value{prim@d}}
			{ \draw[MainSty] (pmain\i) to (pc\i);
			  \begin{scope}[CollSty]
				\draw[->] (pc\i) to (pr\i);
				\draw[->] (pr\i.mid east) to[in=180] (lc\i);
				\draw[->] (lc\i) to[out=0] (pm\i.mid west);
			  \end{scope}
			}
			{	\begin{scope}[MainSty]
				\draw[->] (pmain\i) to[->] (pc\i);
				\draw[->] (pc\i) to (pr\i);
				\draw[->] (pr\i.mid east) to[in=180] (wait);
				\draw[->] (wait) to[out=0] (pm\i.mid west);
				\end{scope}
			}%
			}

		}

		\begin{scope}[xshift=2.0cm,yshift=1cm,start chain=1,node distance=0.875cm,align=center]

			\node [on chain=1,tmtape,draw=none] {\large\textbf{(1)}};
			\node [on chain=1,tmtape,draw=none] {\large\textbf{(2)}};
			\node [on chain=1,tmtape,draw=none] {};
			\node [on chain=1,tmtape,draw=none] {\large\textbf{(3)}};
			\node [on chain=1,tmtape,draw=none] {};
			\node [on chain=1,tmtape,draw=none] {\large\textbf{(4)}};
	
		\end{scope}







\end{scope}

\end{tikzpicture}
%\end{document}

