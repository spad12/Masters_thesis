%% This is an example first chapter.  You should put chapter/appendix that you
%% write into a separate file, and add a line \include{yourfilename} to
%% main.tex, where `yourfilename.tex' is the name of the chapter/appendix file.
%% You can process specific files by typing their names in at the 
%% \files=
%% prompt when you run the file main.tex through LaTeX.
\chapter{Design Options}
\label{ch:design}

	\section{Particle List Structure}
		\subsection{Other Codes}
		\subsection{In house tests}
\noindent \begin{table}[h]
\begin{tabular}{| p{4.0cm} | p{3.5cm} | p{2.5cm} | p{4.0cm} |}
\hline
Component & SoA (ms) & AoS (ms) & Speedup (SoA vs AoS) \\ \hline
Particle data read, move, and write & 758 & 955 & 1.26x \\ \hline
Count Particles & 32.7 & 109 & 3.35x \\ \hline
Data Reorder & 346 & 480 & 1.38x \\ \hline
Total CPU run time & 2491 & 3284 & 1.31x \\ \hline
\end{tabular}
\caption{Execution times of main steps for Array of Structures and Structure of Arrays. Count Particles and Data Reorder are steps used for a sorted particle list. Count Particles counts the number of particles in each sub-domain. Data Reorder reorders the particle list data after the binindex / particle ID pair have been sorted by the radix sort.}
\label{tab:struct_compare} 
\end{table}

	\section{Charge Assign}
		\subsection{Naive Atomic Approach}
		\subsection{Other Codes}

	\section{Particle List Sort}
		\subsection{Costs and Benefits}
		\subsection{Other Codes}
		*Stantchev Particle Binning
		*Kong Particle Passing
		*Linked Particle List
		\subsection{In house tests}

	\section{Particle Advancing}
		\subsection{Assumptions}
		\subsection{Other Codes}
		\subsection{Reinjections and Diagnostics}

	\section{Poisson Solve}
		\subsection{Desired Performance}
		\subsection{Performance vs Implementation Difficulty}

	\section{Grid Dimension Constraints and Handling}
